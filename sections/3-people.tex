\section{People}
\label{section:people}

\subsection{Board and Lead Developers}

The current structure of the \sunpyproj is governed by the \sunpyproj board \citep{christe_steven_2018_3261663}.
The \sunpyproj board is a self-electing oversight board which delegates the majority of the day-to-day operations of the project to a lead developer, who in turn delegates it to members of the community.
The lead developer has overall responsibility for the large scale organization of the \sunpypkg core package, and ensures that pull requests comply with stated standards and align with the goals of the \sunpyproj.
The deputy lead developer supports the lead developer and fills in when the lead is absent.
The board's role is to steer the overall direction of the \sunpyproj and consists of scientists and researchers who are not necessarily involved directly with the day-to-day development of the \sunpypkg core package.

\subsection{Community Roles}
\label{sec:community_roles}

There are several specific community (or executive) roles within the \sunpyproj that perform important duties related to the overall development and maintenance of the project. These roles encompass a range of responsibilities from the development of the core package and affiliated packages to project communication and liaison. The community roles are held by members of the wider solar community who are actively involved in the \sunpyproj. Anyone interested in a community role is encouraged to apply.

At present, there are nine community roles within the \sunpyproj.
From the development side, these include the Lead Developer and the Deputy Lead Developer who are responsible for the development of the \sunpypkg core package, support the development of affiliated packages, and lead the development community.
To assist the Lead/Deputy Developers, there are several development community roles which include:

\begin{itemize}
    \item Continuous Integration Maintainer
    \item Release Manager
    \item Webmaster
    \item Communication and Education Lead who is responsible for the overall engagement with the wider community
    \item Lead Newcomer and Summer of Code mentor who assists new contributors with \sunpyproj development methodologies and oversees the Google Summer of Code project
    \item Affiliated Package Liaison who is responsible for overseeing the affiliated package review process (see \autoref{sec:affiliated_package_application}) and working with developers of current and candidate affiliated packages
\end{itemize}

\subsection{Maintainers and Contributors}

The development of the \sunpypkg core package depends principally on an established team of volunteers from the development community that support the Lead and Deputy Lead Developers.
These volunteer \textit{maintainers} are given commit access to the \sunpypkg repository.
Those who maintain the \sunpypkg core package are predominantly, though not exclusively, scientists from the solar community who use \sunpypkg in their work.
In addition to this group of core maintainers, there is a steady influx of new \textit{contributors}, averaging around 20--25 people per year.
These contributors enable a wider range of features and code changes than would otherwise be normally possible due to the time constraints of the established team of volunteers.
Within this subset of maintainers are members who maintain the specific sub-packages within \sunpypkg like Map or Coordinates.
These individuals are selected due to either their specific knowledge of the topic or their expertise with these sub-packages.

Contributing to the \sunpyproj includes a wide range of activities, not all of them programming related.
These include reporting bugs by raising issues on GitHub, requesting features, writing code and tests, providing feedback on pull requests, correcting or adding documentation, helping people who have problems or questions in communication channels and more.
The \sunpyproj is always looking for any new volunteers or people willing to contribute their time.
