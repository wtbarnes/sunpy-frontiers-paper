\section{Introduction}
\label{sec:introduction}

The \sunpyproj is an organization whose mission is to develop and facilitate a high-quality, easy-to-use, community-led, free and open-source solar data analysis ecosystem based on the scientific Python environment.
The vision of the project is to build a diverse and inclusive solar physics and heliophysics community that supports scientific discovery and \added[id=WTB]{enables} reproducibility through the development of accessible, open-source software \citep{bobra_monica_2020_7020094}.
To achieve this mission and to make this vision a reality, the \sunpyproj maintains and guides the development of a number of Python packages including the \sunpypkg core package, and organizes educational activities around the use of Python for solar-physics research.

As the scientific Python environment matured in the early 2010s \citep{Hunter:2007, harris2020array, 2020SciPy-NMeth}, the development of a Python package devoted to solar physics became viable.
This led to the founding of the \sunpyproj led by scientists at NASA Goddard Space Flight Center in 2011.
The goal of the \sunpyproj at that time was to develop a package that provided the core functionality needed for solar data analysis in Python.
To distinguish the software package from the wider project, this original package is now known as the \sunpypkg core package \citep{sunpy_community2020}.
As the \sunpyproj and \sunpypkg grew, an ecosystem of affiliated packages (see \autoref{ssec:affiliated-packages}) was developed to keep the \sunpypkg core package from becoming too large and difficult to manage.

The \sunpyproj is committed to the principles of open development.
All code is hosted and openly-developed on GitHub\footnote{\url{https://github.com/sunpy/sunpy}} in order to enable anyone to contribute code or provide feedback.
All packages within the \sunpyproj must be under an Open Source Initiative (OSI)\footnote{\url{https://opensource.org}} approved license.
Discussion is hosted on several open communication channels which include weekly community calls, mailing lists, a Discourse forum, and instant messaging via Matrix\footnote{\url{https://matrix.org/}}.
Additionally, the \sunpyproj has a code of conduct\footnote{\url{https://sunpy.org/coc}} to ensure that communication within the project is open, considerate, and respectful.

The aim of this paper is to give a high level description of the \sunpyproj, including its various components, and to describe the direction of the project in the coming years.
\autoref{sec:code} describes the various Python packages that form the project, including both the \sunpypkg core package (\autoref{ssec:the-sunpypkg-core-package}) and the various affiliated packages (\autoref{ssec:affiliated-packages}).
\autoref{sec:people} gives an overview of the roles within the project and describes how to become involved with \sunpyproj.
\autoref{sec:community} describes the various activities of the project within the broader solar physics community.
Finally, \autoref{sec:the-future-of-the-sunpyproj} lays out a vision for the future of the \sunpyproj.
